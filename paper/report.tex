\documentclass{article}

\usepackage[final]{neurips_2022}

\usepackage[utf8]{inputenc}
\usepackage[T1]{fontenc}
\usepackage{hyperref}
\usepackage{url}
\usepackage{booktabs}
\usepackage{amsfonts}
\usepackage{nicefrac}
\usepackage{float}
\usepackage{amsmath}
\usepackage{algorithm}
\usepackage{algpseudocode}
\usepackage{microtype}
\usepackage{xcolor}
\usepackage{bm}
\usepackage{graphicx}
\usepackage{subfigure}
\usepackage{listings}
\usepackage{braket}

\hypersetup{
  colorlinks=true,
  linkbordercolor=white,
}

\renewcommand{\algorithmicrequire}{\textbf{Input:}}
\renewcommand{\algorithmicensure}{\textbf{Output:}}

\title{Explicit Quantum Circuits for Block Encodings of Sparse Matrices: A Review and Exploration}

\author{%
  \large Zibo Ren \\
  \large \texttt{2200010626}
  \And
  \large Zixuan Yuan \\
  \large \texttt{2200010825}
}

\begin{document}

\large

\maketitle

\begin{abstract}

  Efficiently constructing quantum circuits for block encodings of matrices is crucial for leveraging quantum linear algebra algorithms, which promise significant speedups for many computational problems.
  A block encoding is a technique where a matrix of interest is embedded within a larger unitary matrix, making it amenable to quantum computation.
  The realization of quantum advantage, however, heavily relies on the effective construction of these block encoding circuits, a task that presents considerable challenges, even when dealing with well-structured sparse matrices.
  This report reviews the work by Camps et al. \cite{EQC} on explicit circuit constructions for block encodings of certain well-structured sparse matrices.
  We delve into their proposed strategies, analyze their numerical demonstrations and reproduce their experiments.
  Furthermore, we explore potential avenues for original contributions and future research directions stemming from this work.
  We also provide implementations of the quantum circuits discussed in this paper in Python.

\end{abstract}

\section{Introduction}

Quantum linear algebra algorithms have emerged as a promising avenue to achieve exponential speedups over classical counterparts in solving fundamental computational problems such as solving linear systems, eigenvalue decomposition, and singular value transformation \cite{EQC}. At the heart of these algorithms lies the technique of \emph{block encoding}, which enables the embedding of non-unitary matrices into larger unitary matrices\textemdash a prerequisite for quantum computation. Formally, an $(\alpha, m, \varepsilon)$-block-encoding of a matrix $A \in \mathbb{C}^{N \times N}$ involves constructing an $(m+n)$-qubit unitary $U_A$ such that
$$\|A - \alpha(\bra{0^m} \otimes I_N)U_A(\ket{0^m} \otimes I_N)\|_2 \leq \varepsilon$$

where $N=2^n$. However, despite theoretical guarantees of algorithmic efficiency, practical implementations of block encodings remain challenging, particularly for structured sparse matrices\textemdash a critical class of inputs for applications in graph theory, differential equations, and machine learning.

Prior work on block encoding primarily focused on abstract oracle-based constructions, leaving explicit circuit designs largely unexplored. For instance, while quantum singular value transformation (QSVT) \cite{Gilyen2019} theoretically enables polynomial transformations of matrices via block encodings, its practical utility hinges on the explicit construction of quantum circuits for structured matrices. Camps et al. \cite{EQC} recently addressed this gap by introducing a systematic framework for constructing efficient quantum circuits for $s$-sparse matrices, which are matrices with at most $s$ nonzeros per column. Their approach decomposes the block encoding unitary $U_A$ into two components: (1) an oracle circuit $O_C$ that encodes the sparsity pattern of $A$, and (2) an oracle circuit $O_A$ that encodes the numerical values of nonzero entries. This explicit decomposition bridges the gap between theoretical algorithms and hardware-realizable implementations.

The significance of this work is twofold. First, for scaled $s$-sparse matrices $A/s$, the authors demonstrate circuits with gate complexity $\text{poly}(n)$, where $n$ is the qubit count, by leveraging controlled rotations and shift operators (e.g., implementing the mapping $\ket{j} \mapsto \ket{\text{mod}(j \pm 1, N)}$ for cyclic graph adjacency matrices) \cite{EQC}. Second, they extend their framework to symmetric stochastic matrices, enabling direct block encodings of Chebyshev polynomials $T_k(P)$ for quantum walks\textemdash a task previously hindered by scaling factors $1/s$ that degraded algorithmic efficiency. For example, their explicit construction of a quantum circuit for the block encoding of a $8 \times 8$ circulant matrix achieves optimal depth by combining Hadamard gates and controlled-$R$ shifts (Fig. 7 in \cite{EQC}).

This review synthesizes the key contributions of Camps et al. \cite{EQC}, including their general strategy for sparse matrix block encoding, numerical demonstrations using the QCLAB toolbox in MATLAB, and connections to quantum walk algorithms. We further explore extensions of their work, such as applications to machine learning tasks like quantum principal component analysis, and provide Python implementations of their circuits for reproducibility. By analyzing both theoretical and practical aspects, this report aims to illuminate pathways for advancing quantum linear algebra algorithms in real-world applications.

\section{Related Works}



\bibliographystyle{unsrt}
\bibliography{references}
\end{document}