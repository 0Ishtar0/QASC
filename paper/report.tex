\documentclass{article}

\usepackage[final]{neurips_2022}
\usepackage[UTF8]{ctex}

\usepackage[utf8]{inputenc}
\usepackage[T1]{fontenc}
\usepackage{hyperref}
\usepackage{url}
\usepackage{booktabs}
\usepackage{amsfonts}
\usepackage{nicefrac}
\usepackage{float}
\usepackage{amsmath}
\usepackage{algorithm}
\usepackage{algpseudocode}
\usepackage{microtype}
\usepackage{xcolor}
\usepackage{bm}
\usepackage{graphicx}
\usepackage{subfigure}
\usepackage{listings}

\hypersetup{
    colorlinks=true,
    linkbordercolor=white,
}

\renewcommand{\algorithmicrequire}{\textbf{Input:}}
\renewcommand{\algorithmicensure}{\textbf{Output:}}


\title{Explicit Quantum Circuits for Block Encodings of Sparse Matrices: A Review and Exploration}

\author{%
    \large Zibo Ren \\
    \large \texttt{2200010626}
    \And
    \large Zixuan Yuan \\
    \large \texttt{2200010825}
}



\begin{document}


    \maketitle

    \begin{abstract}

        Efficiently constructing quantum circuits for block encodings of matrices is crucial for leveraging quantum linear algebra algorithms, which promise significant speedups for many computational problems.
        A block encoding is a technique where a matrix of interest is embedded within a larger unitary matrix, making it amenable to quantum computation.
        The realization of quantum advantage, however, heavily relies on the effective construction of these block encoding circuits, a task that presents considerable challenges, even when dealing with well-structured sparse matrices.
        This report reviews the work by Camps et al. \cite{EQC} on explicit circuit constructions for block encodings of certain well-structured sparse matrices.
        We delve into their proposed strategies, analyze their numerical demonstrations and reproduce their experiments.
        Furthermore, we explore potential avenues for original contributions and future research directions stemming from this work.
        We also provide implementations of the quantum circuits discussed in this paper in Python.

    \end{abstract}

    \bibliographystyle{unsrt}
    \bibliography{references}
\end{document}